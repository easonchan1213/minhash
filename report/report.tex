%%%%%%%%%%%%%%%%%%%%%%%%%%%%%%%%%%%%%%%%%%%%%%%%%%%%%%%%%%%%%%%%%%%%%%
% LaTeX Example: Project Report
%
% Source: http://www.howtotex.com
%
% Feel free to distribute this example, but please keep the referral
% to howtotex.com
% Date: March 2011 
% 
%%%%%%%%%%%%%%%%%%%%%%%%%%%%%%%%%%%%%%%%%%%%%%%%%%%%%%%%%%%%%%%%%%%%%%
% How to use writeLaTeX: 
%
% You edit the source code here on the left, and the preview on the
% right shows you the result within a few seconds.
%
% Bookmark this page and share the URL with your co-authors. They can
% edit at the same time!
%
% You can upload figures, bibliographies, custom classes and
% styles using the files menu.
%
% If you're new to LaTeX, the wikibook is a great place to start:
% http://en.wikibooks.org/wiki/LaTeX
%
%%%%%%%%%%%%%%%%%%%%%%%%%%%%%%%%%%%%%%%%%%%%%%%%%%%%%%%%%%%%%%%%%%%%%%
% Edit the title below to update the display in My Documents
%\title{Project Report}
%
%%% Preamble
\documentclass[paper=a4, fontsize=11pt]{scrartcl}
\usepackage[T1]{fontenc}
\usepackage{fourier}
\usepackage{hyperref}

\usepackage[english]{babel}															% English language/hyphenation
\usepackage[protrusion=true,expansion=true]{microtype}	
\usepackage{amsmath,amsfonts,amsthm} % Math packages
\usepackage[pdftex]{graphicx}	
\usepackage{url}


%%% Custom sectioning
\usepackage{sectsty}
\allsectionsfont{\centering \normalfont\scshape}


%%% Custom headers/footers (fancyhdr package)
\usepackage{fancyhdr}
\pagestyle{fancyplain}
\fancyhead{}											% No page header
\fancyfoot[L]{}											% Empty 
\fancyfoot[C]{}											% Empty
\fancyfoot[R]{\thepage}									% Pagenumbering
\renewcommand{\headrulewidth}{0pt}			% Remove header underlines
\renewcommand{\footrulewidth}{0pt}				% Remove footer underlines
\setlength{\headheight}{13.6pt}


%%% Equation and float numbering
\numberwithin{equation}{section}		% Equationnumbering: section.eq#
\numberwithin{figure}{section}			% Figurenumbering: section.fig#
\numberwithin{table}{section}				% Tablenumbering: section.tab#


%%% Maketitle metadata
\newcommand{\horrule}[1]{\rule{\linewidth}{#1}} 	% Horizontal rule

\title{
		%\vspace{-1in} 	
		\usefont{OT1}{bch}{b}{n}
		\normalfont \normalsize \textsc{Dawanda Challenge} \\ [25pt]
		\horrule{0.5pt} \\[0.4cm]
		\huge Recommendation Engine using MinHash \\
		\horrule{2pt} \\[0.5cm]
}
\author{
		\normalfont 								\normalsize
        Lorenzo Fundaro\\[-3pt]		\normalsize
        \today
}
\date{}


%%% Begin document
\begin{document}
\maketitle

\section{Problem Modeling}
\subsection{Data Structures}
\paragraph{}
The problem of finding similarity between sets can be tackled using a technique called MinHash. For that purpose, it is common 
to model all the sets and the elements they have in a \textit{characteristic matrix} shown as follows:

\begin{table}[h]
\centering
\begin{tabular}{|c|c|c|c|c|}
\hline
      & $S_1$ & $S_2$ & ... & $S_n$ \\ \hline
$e_1$ & 1     & 0     & 0   & 1     \\ \hline
$e_2$ & 0     & 0     & 1   & 1     \\ \hline
...   &       &       &     &       \\ \hline
$e_n$   & 1     & 0     & 1   & 1     \\ \hline
\end{tabular}
\end{table}

where ($e_i$,$S_i$) = 1 if $S_i$ contains the element $e_i$ and ($e_i$,$S_i$) = 0 otherwise. This matrix is naturally very \textit{sparse}, so for 
this particular implementation I decided to use a map where the keys represent a \textit{productId} and the correspondent value is a list of integers containing several 
\textit{userId}'s. This decision would allow for easier traversal when calculating the signature matrix that MinHash requires to build.

\subsection{Hashing Functions}

For the set of hashing functions I created a class called \textit{Hash} under the package \textit{utils}.
This class is useful for generating a list of functions that are applied to every element of the \textit{signature matrix}. The following 
function is used for hashing:

\begin{align} 
	h &= (a_i*x + b_i) \bmod 2147483647					
\end{align}

where ${a_i,b_i}$ are coefficients generated randomly and the large number is a prime number. 


\section{Problems encountered during development}

\begin{itemize}
\item \textbf{Complex traversal on data structure:} when applying the MinHash algorithm one has to make calculations with the signature matrix. 
For this calculations, it is necessary to traverse the list of products and for each product grab the list of users that have bought it and 
for each user hash its column in the signature matrix and save the changes for the next traversal. In functional programming side effects
are not allowed, but in this case, to make things easier I decided to make the signature matrix a variable and not a value. 
\item \textbf{Hash function:} in order to get recommendations, the amount of hashing functions should be between 4 and 10, any number 
beyond that range would not allow for any hash collisions, thus giving no output. I still don't understand very well why I cannot use, e.g.: 100 functions. 
Choosing the right hashing function is still an open question for me. 
\end{itemize}

\section{Possible Improvements}

Once we have calculated the signature matrix we can proceed on the suggestion of products for a given user. This calculation involves 
comparing the signature column of the user with the rest of the users and obtaining the similarity index. The majority of these calculations 
don't throw any results and a lot of CPU time is wasted for bigger datasets. By using a technique called Locality Sensitive Hashing (LSH) we can 
reduce the candidate sets considering only those signatures columns that when compared will give a positive result.

\section{Resources}

\begin{itemize}
\item Mining of Massive Datasets by Jure Leskovec, Anand Rajaraman and Jeff Ullman.
\item Google News Personalization: Scalable Online Collaborative Filtering.
\item \href{http://www.tonicebrian.com/2013/03/11/introduction-to-the-minhash-algorithm/}{Introduction to the MinHash algorithm}
\end{itemize}

%%% End document
\end{document}